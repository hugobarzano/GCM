
\vspace{0.7cm}
\noindent{\textbf{Resumen}}\\


Este proyecto demuestra como el uso de conceptos ligados a la  Programación Generativa y la integración de diversos servicios web pueden utilizarse de manera conjunta para sintetizar, generalizar y automatizar el proceso de desarrollo y operación de aplicaciones software. La lectura aquí propuesta, junto con el sistema software resultante abordan  el proceso de desarrollo de aplicaciones software mediante un enfoque generativo, con esto  se pretende proporcionar al desarrollador software de un sistema web orientado a la generación, construcción y despliegue de aplicaciones software, concretamente de aplicaciones  web o servicios de datos en la nube.
 
 En este proyecto se considera que las etapas que forman el ciclo de vida o proceso de desarrollo de aplicaciones software está altamente segmentado, por lo que presenta una solución generativa y automatizada al proceso de desarrollo software. Con este proyecto no se pretende desprestigiar a los procesos de desarrollo software tradicionales, lo que intenta es demostrar como la unificación de los conceptos comunes entre aplicaciones web de diversa naturaleza así como la síntesis y  automatización genérica de las etapas básicas de producción software pueden dar lugar a soluciones funcionales ahorrando al desarrollador numerosas horas de trabajo. De igual forma, el sistema resultante  puede ser aprovechado por el desarrollador inexperto como punto de partida en su aprendizaje de las denominadas tecnologías de la información.
 
 
Este proyecto pone de manifiesto que ante la gran diversidad de tecnologías y soluciones ofrecidas por la Cloud Computing  existen definiciones, conceptos y  etapas básicas que son  comunes  y transversales  a todas ellas, independientemente de su naturaleza, dominio y propósito. Este proyecto sintetiza el proceso de desarrollo software en 3 etapas fundamentales: 

\begin{enumerate}
\item \textbf{ Code Generation. }  Etapa inicial donde se sintetiza el proceso de desarrollo software, es decir, se produce el código necesario para la solución especificada por el desarrollador. 
\item \textbf{ Artefact Generation, }  El objetivo de esta etapa es aprovechar el código fuente generado en la etapa Code Generation para producir un artefacto auto-contenido junto con la configuración necesaria para su ejecución en un entorno virtual. 
\item \textbf{ Deploy Generation. } En esta etapa se aprovecha el producto de la etapa Artefact Generation para disponibilizar la solución en Internet. 
\end{enumerate}
 

En los siguientes capítulos se profundiza en estas ideas, detallando cuales son los objetivos de alto nivel o casos de uso que se esperan del sistema. Se describirá el problema planteado y se analizará en detalle la solución alcanzada.\\ \\

\noindent{\textbf{Palabras Clave}: Cloud Computing, Programación Generativa, Desarrollo Software, Código Fuente, Automatización, Integración, Despliegue, Aplicaciones, Servicios, Contenedores.}\\

\afterpage{\null\newpage}
\newpage

