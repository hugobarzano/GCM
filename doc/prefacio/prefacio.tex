
\vspace{0.7cm}
\noindent{\textbf{Resumen}}\\


Este proyecto muestra como el uso de la programación generativa y las tecnologías de la información mejoran el proceso  aprendizaje y desarrollo software en el ámbito de las soluciones orientadas a la web 2.0. 

Se pretende proporcionar al desarrollador software de un sistema web orientado a la generación, construcción y despliegue de aplicaciones web o servicios cloud  de diversa naturaleza partiendo de una especificación de alto nivel.  Normalmente, las etapas de desarrollo software están segmentadas en el desarrollo del código, la compilación de este código junto con las dependencias de terceros necesarias para producir un ejecutable, binario o artefacto para su posterior ejecución,  la configuración necesaria para poder ejecutar el artefacto resultante así como  la infraestructura virtual necesaria para desplegar y soportar la actividad de negocio de  dicho software, siendo este accesible desde Internet. 

Con este proyecto no se pretende desprestigiar a los procesos de desarrollo software tradicionales pero si mostrar como la unificación de los conceptos comunes entre aplicaciones web de diversa naturaleza así como la automatización genérica de las etapas básicas de producción software pueden dar lugar a soluciones funcionales ahorrando al desarrollador numerosas horas de trabajo. Otro enfoque igual de valido es el de dotar al desarrollador inexperto de un punto de partida en su aprendizaje. 


Este proyecto pone de manifiesto que independientemente del tipo de aplicación web o servicio cloud que el desarrollador quiera producir existen unas etapas básicas, comunes  y transversal por las que la gran mayoría  aplicaciones han de pasar comenzando por la implementación del código fuente hasta alcanzar la madurez necesaria para ser accesibles a través de Internet.  En este proyecto dichas etapas se sintetizan de la siguiente manera: 

\begin{enumerate}
\item \textbf{ Code Generation }  Etapa inicial donde se sintetiza el proceso de desarrollo software, es decir se produce el código necesario para la solución especificada por el desarrollador. 
\item \textbf{ Artefact Generation }  El objetivo de esta etapa es aprovechar el código fuente generado en la etapa Code Generation para producir un artefacto auto-contenido junto con la configuración necesaria para su ejecución en un entorno virtual. 
\item \textbf{ Deploy Generation } En esta etapa se aprovecha el producto de la etapa Artefact Generation para disponibilizar la solución en Internet. 
\end{enumerate}


En los siguientes capítulos se profundiza en estas ideas, detallando cuales son los objetivos de alto nivel o casos de uso que se esperan del sistema así como se entrará en detalle de cada una de estas etapas.  


