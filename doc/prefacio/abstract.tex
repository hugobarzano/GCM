
\vspace{0.7cm}
\noindent{\textbf{Abstract}}\\


This project shows how using concepts of  Generative Programming and the integration of web services can work toogether to synthesize, generalize and automate the process of  software development and software operation. This proyecct  addresses the process of developing software applications through a generative approach, with the target of provide a web system oriented to software generation, artefact construction and applicatiioons deployment, specifically from web applications or cloud data services.

 
This project considers that the stages of software applications developements are highly segmented, so this proyect presents a generative and automated solution to the software development. This project is not intended to discredit traditional software development but rather to show how the unification of common concepts among web applications kinds as well as the generic synthesis and automation of the basic stages of software production can lead to functional solutions saving developers from larges amounts of work. In the same line, the system can be used by the juniors developers as a starting point in their learning path of the information technologies.
 
 
This project shows that given the great diversity of technologies and solutions offered by Cloud Computing, there are definitions, concepts and basic stages that are common and transversal to all of them, regardless of their nature, domain or purpose. This project synthesizes the software development process in 3 fundamental stages:

\begin{enumerate}
\item \textbf{ Code Generation. }  Initial stage where the software development process is synthesized. Source code necessary for the solution specified by the developer is produced.

\item \textbf{ Artefact Generation }  The goal of this stage is to use the source code generated in the Code Generation stage to produce a self-contained artifact along with the necessary setup for its execution in a virtual environment.

\item \textbf{ Deploy Generation } Final stage where the product of the Artefact Generation stage is used to run the solution and make it available to developers over Internet.
\end{enumerate}
 

The following chapters delve into these ideas, describing project objectives and use cases expected from the system. The problem will be officially presented and the solution reached will be analyzed in detail.
